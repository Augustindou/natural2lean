Theorem Square mod 3:
If $q$ is not divisible by $3$, then $q^2 \mod 3 = 1$.


Considering the different possibilities of the modulo operation, we have $q \mod 3 \in \{ 0, 1, 2 \}$
As $q$ is not divisible by $3$, we have a contradiction for $0$.
For the second case, we have $q \mod 3 = 1$
    $q$ can be rewritten as $q = 3k + 1$ with $k \in \mathcal{N}$
    By development, we have therefore $q^2 = (3k+1)^2 = 9k^2+6k+1 = 3(3k^2 + 2k) + 1$
    hence, using $3k^2 + 2k$, we have $q^2 \mod 3 = 1$
For the last case, $q \mod 3 = 2$
    Following the same development, we have $q^2 = (3k+2)^2 = 3(3k^2+4k+1) +1$
    Hence, we have $q^2 \mod 3 = 1$ again
QED